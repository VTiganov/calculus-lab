\textbf{Цель лабораторной работы}
\vspace{0.5cm}

Цель работы заключалась в следующем: ознакомиться с критериями унимодальности функции, оценкой экстремума без определения производной функции и решением задачи нахождения корней на заданном отрезке, применить полученные знания на практике для выполнения заданий лабораторной работы. 


\vspace{2cm}

\textbf{Задачи лабораторной работы}
\vspace{0.5cm}

C помощью выбранного языка программирования реализовать функции для двух заданий:
\begin{enumerate}
    \item Реализовать поиск экстремума функции, имея только метод определения \textit{f(x)} при заданном \textit{x}. Был выбрал метод Дихотомии. Про него с теоретической и практической стороны см. далее.
    \item Реализовать программу, которая решит уравнение \textit{f(x) = 0, $x\in$[a,b], f(x)} определена и непрерывна на \textit{[a,b]}. Проанализировать отклонения и обработать исключения.
    \item Провести ручное и автоматическое тестирование программы, убедиться в корректности работы.
\end{enumerate}