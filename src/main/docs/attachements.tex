\begin{center}
    ПРИЛОЖЕНИЯ\\
    \textit{Все еще не научился писать в листинге на русском, переведено с помощью ChatGPT}
\end{center}

\begin{lstlisting}[style=JavaStyle, caption={Java Code for Task1}]
package com.example;

public class Task1 {

    // Calculates the value of the given expression for a specific x
    public static float func(String expression, float x) {
        Expression e = new ExpressionBuilder(expression) // Build the mathematical expression using exp4j
                .variable("x")       // Define the variable "x"
                .build()             // Compile the expression
                .setVariable("x", x); // Set the value of "x"
        return (float) e.evaluate();   // Evaluate and return the result
    }

    // Implements the dichotomy method to find the root of the equation
    public static float dichotomyMethod(String expression, float a, float b, float tolerance) {
        if (func(expression, a) * func(expression, b) > 0) {
            // Check if the root exists within the interval [a, b]
            throw new IllegalArgumentException("f(a) and f(b) must have opposite signs!");
        }

        float c; // Midpoint of the interval
        XYSeries series = new XYSeries("Iterations"); // Create a series to store iteration data
        ArrayList<Float> intervalLengths = new ArrayList<>(); // List to store interval lengths

        // Continue iterating until the interval size is less than the tolerance
        while ((b - a) / 2.0 > tolerance) {
            c = (a + b) / 2;  // Compute the midpoint
            series.add(c, func(expression, c));  // Add the midpoint and its function value to the series

            intervalLengths.add((b - a));   // Record the current interval length

            if (func(expression, c) == 0) {   // Check if an exact root is found
                return c;
            } else if (func(expression, c) * func(expression, a) < 0) {
                b = c;   // Update the upper bound
            } else {
                a = c;  // Update the lower bound
            }
        }

        series.add((a + b) / 2, func(expression, (a + b) / 2)); // Add the final midpoint

        showGraph(series); // Display the graph of the function values
        plotIntervalLengths(intervalLengths); // Display the interval length convergence graph

        return (a + b) / 2; // Return the approximate root
    }

    // Displays a graph of the function values across iterations
    public static void showGraph(XYSeries series) {
        SwingUtilities.invokeLater(() -> {
            XYSeriesCollection dataset = new XYSeriesCollection(series); // Create a dataset for the series

            JFreeChart chart = ChartFactory.createXYLineChart(
                    "Estimated position of extremum", // Title of the chart
                    "x",   // Label for the X-axis
                    "f(x)",   // Label for the Y-axis
                    dataset,   // Data for the chart
                    PlotOrientation.VERTICAL,   // Chart orientation
                    true,   // Show legend
                    true,   // Enable tooltips
                    false                              
            );

            XYPlot plot = chart.getXYPlot();
            XYLineAndShapeRenderer renderer = new XYLineAndShapeRenderer(true, true); // Configure renderer for lines and shapes

            renderer.setSeriesShape(0, new java.awt.geom.Ellipse2D.Double(-6, -6, 10, 10)); // Set point shapes
            renderer.setSeriesPaint(0, Color.BLACK); // Set line color
            renderer.setSeriesStroke(0, new BasicStroke(1.0f)); // Set line thickness

            plot.setRenderer(renderer);

            ChartPanel chartPanel = new ChartPanel(chart); // Create a chart panel
            chartPanel.setPreferredSize(new java.awt.Dimension(800, 600)); // Set panel size

            JFrame frame = new JFrame("Graph"); // Create a JFrame to hold the chart
            frame.getContentPane().add(chartPanel, BorderLayout.CENTER); // Add chart panel to frame
            frame.pack(); // Adjust frame size to fit content
            frame.setVisible(true); // Display the frame
        });
    }

    // Plots the interval lengths across iterations
    public static void plotIntervalLengths(ArrayList<Float> intervalLengths) {
        SwingUtilities.invokeLater(() -> {
            XYSeries intervalSeries = new XYSeries("Interval Lengths"); // Create a series for interval lengths
            for (int i = 0; i < intervalLengths.size(); i++) {
                intervalSeries.add(i + 1, intervalLengths.get(i)); // Add each interval length with its iteration index
            }

            XYSeriesCollection dataset = new XYSeriesCollection(intervalSeries); // Create a dataset for the series
            JFreeChart chart = ChartFactory.createXYLineChart(
                    "Interval Length vs Iteration", 
                    "Iteration",                    
                    "Interval Length",              
                    dataset,                         
                    PlotOrientation.VERTICAL,        
                    true,
                    true,   
                    false                        
            );

            XYPlot plot = chart.getXYPlot();
            XYLineAndShapeRenderer renderer = new XYLineAndShapeRenderer(true, false);
            renderer.setSeriesStroke(0, new BasicStroke(5.0f));

            plot.setRenderer(renderer);

            ChartPanel chartPanel = new ChartPanel(chart);
            chartPanel.setPreferredSize(new java.awt.Dimension(800, 600)); 

            JFrame frame = new JFrame("Interval Lengths"); 
            frame.getContentPane().add(chartPanel, BorderLayout.CENTER);
            frame.pack(); 
            frame.setVisible(true); 
        });
    }
}

    \end{lstlisting}
    
    \begin{lstlisting}[style=JavaStyle, caption={Java Code for Task2}]
package com.example;

public class Task2 {

    public static List<Double> findRoots(String function, double a, double b, double epsilon) {
        Expression expression = new ExpressionBuilder(function)
                .variables("x")
                .build();

        List<Double> roots = new ArrayList<>();
        double step = epsilon; // Step size for checking intervals

        for (double x = a; x <= b; x += step) {
            if (f(expression, x) * f(expression, x + step) < 0) {
                double root = findRootInInterval(expression, x, x + step, epsilon);
                roots.add(root);
            }
        }

        return roots;
    }

    private static double findRootInInterval(Expression expression, double a, double b, double epsilon) {
        if (f(expression, a) * f(expression, b) >= 0) {
            return Double.NaN;
        }

        double c = a;
        int iterations = 0;

        while ((b - a) >= epsilon) {
            c = (a + b) / 2;
            if (f(expression, c) == 0.0) {
                break;
            } else if (f(expression, c) * f(expression, a) < 0) {
                b = c;
            } else {
                a = c;
            }
            iterations++;
        }

        System.out.println("Number of iterations: " + iterations); // Print iteration count
        return c;
    }

    private static double f(Expression expression, double x) {
        expression.setVariable("x", x); // Set the variable 'x' in the expression
        return expression.evaluate(); // Evaluate the expression and return the result
    }

    public static double calculateRMS(double foundRoot, double trueRoot) {
        return Math.sqrt((foundRoot - trueRoot) * (foundRoot - trueRoot)); // Calculate RMS error
    }
}

    \end{lstlisting}

\begin{lstlisting}[style=JavaStyle, caption={Used Libraries}]
import net.objecthunter.exp4j.Expression; // For evaluating mathematical expressions
import net.objecthunter.exp4j.ExpressionBuilder; // For building expressions with variables
import org.jfree.chart.ChartFactory; // For creating charts
import org.jfree.chart.ChartPanel; // For embedding charts in Swing applications
import org.jfree.chart.JFreeChart; // Represents a chart in JFreeChart library
import org.jfree.chart.plot.PlotOrientation; // Defines the orientation of a plot (horizontal or vertical)
import org.jfree.data.xy.XYSeries; // For storing series of XY data for plotting
import org.jfree.data.xy.XYSeriesCollection; // A collection of XYSeries for plotting

import javax.swing.*; // For GUI components like JFrame, JPanel, etc.
import java.awt.*; // For handling graphics and layout management
import org.jfree.chart.plot.XYPlot; // For managing the plotting of XY data
import org.jfree.chart.renderer.xy.XYLineAndShapeRenderer; // For customizing the appearance of XY plots
import java.awt.BasicStroke; // For setting line styles in graphics
import java.util.ArrayList; // For creating dynamic arrays (lists)
        \end{lstlisting}
        