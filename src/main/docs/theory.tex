\textbf{Теоретическая подготовка}
\vspace{0.5cm}

Использованные средства и инструменты для выполнения лабораторной работы:
\begin{itemize}
    \item Использовали язык программирования Java.
    \item Использованы стандартные Java библиотеки для ввода математической функции от пользователя (для тестов и прочего) и построения графиков.
    \item Использованы стандартные библиотеки для написания \textit{Unit-тестов.}
    \item Работа проводилась совместно с помощью \textit{Github}, задачи были распределены между участниками.
    \item Весь проект был собран с помощью сборщика $Maven$ для удобного запуска и тестирования на любом устройстве с $JVM$ (виртуальная машина $Java$, для запуска скомпилированного $.jar$ архива).
\end{itemize}
\vspace{1cm}

\textbf{Теоретическая подготовка, метод Дихотомии}\\

Метод Дихотомии для определения экстремума функции (по совместительству и для нахождения единственного корня)
на заданном промежутке, если функция в конечных точках этого промежутка имеет разные знаки, определена на всем промежутке и непрерывна 
на нем. (согласно теореме о промежуточном значении: если на отрезке функция имеет разные знаки, то где-то между
этими двумя точками существует хотя бы один корень функции)\\

Метод чем-то напоминает бинарный поиск. Суть его работы алгоритмически:
\begin{enumerate}
    \item Определяется промежуток, на котором требуется найти корень. Предполагается, что условия смены знака и непрерывности выполнены.
    \item Вычисляется середина отрезка с помощью формулы \( m = \frac{a + b}{2} \). 
    \item Если \( f(m) = 0 \), то корень найден. 
    \item Если \( f(m) \cdot f(a) < 0 \), то корень лежит в интервале \([a, m]\), и обновляем правую границу: \(b = m\).
    \item Если \( f(m) \cdot f(b) < 0 \), то корень лежит в интервале \([m, b]\), и обновляем левую границу: \(a = m\).
    \item Процесс повторяется, пока длина интервала не станет достаточно малой, что означает приближённое значение корня.
\end{enumerate}
\textit{В программе реализован ручной ввод пользователем математической функции и границ, в которых требуется изучить функцию и найти приблизительное значение экстремума.} \\
\\
Согласно техническому заданию, предполагаемый экстремум на каждой итерации добавляется на график,
строится график сужения интервала поиска после каждой итерации, выводится график со всеми точками
предполагаемых экстремумов и динамики уменьшения зоны поиска.
\vspace{1cm}

\textbf{Теоретическая подготовка, решение уравнения вида \textit{f(x) = 0}}\\ 
Требовалось реализовать программу, которая решит следующую задачу:
Для заданной функции $f(x)$, гарантированно непрерывной и определенной на интервале
$[a,b]$ найти решение уравнения $f(x) = 0$.\\
При этом реализован функционал:
ввод пользователем интересующих границ для изучения функции, ввод заведомо верного корня,
подсчет среднеквадратичного отклонения найденного решения по отношению к верному. (Предполагается,
что оно известно заранее и введено корректно)
\\

По большому счету, данная программа представляет собой доработанную версию предыдущей, в
которой был использован принцип дихотомии.

Произведенные улучшения:
\begin{itemize}
    \item Обработка случаев с несколькими корнями.
    \item Вывод нескольких корней и расчет отклонения для каждого из них.
    \item Согласно техническому заданию, обработка случая, когда корней нет. (реализовано и протестировано также в первом)
\end{itemize}

Суть работы алгоритмически:
\begin{enumerate}
    \item Имеем зону поиска и требуемую точность.
    \item Определяем, есть ли корни вообще:
    \item Итерируемся через всю зону изначального поиска, проверяя концы на различие в знаках.
    \item Если знаки различаются, корень есть, ищем его и добавляем в список корней. \begin{center}
        $f(x) \cdot f(x + \epsilon) < 0$, $x \in [a,b]$, $\epsilon > 0$
    \end{center}
    \item Повторяем итерирование до конца отрезка. \begin{center}
        $x + \epsilon \leq b, x \in [a,b]$
    \end{center}
\end{enumerate}