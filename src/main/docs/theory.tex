\textbf{Теоретическая подготовка}
\vspace{0.5cm}

Использованные средства и инструменты для выполнения лабораторной работы:
\begin{itemize}
    \item Использовали язык программирования Java.
    \item Использованы стандартные Java библиотеки для ввода математической функции от пользователя (для тестов и прочего) и построения графиков.
    \item Использованы стандартные библиотеки для написания \textit{Unit тестов.}
    \item Работа проводилась совместно с помощью \textit{Github}, задачи были распределены между участниками.
\end{itemize}
\vspace{1cm}

\textbf{Теоретическая подготовка, метод Дихотомии}\\

Метод Дихотомии для определения экстремума функции (по совместительству и для нахождения единственного корня)
на заданном промежутке, если функция в конечных точках этого промежутка имеет разные знаки, определена на всем промежутке и непрерывна 
на нем. (согласно теореме о промежуточном значении: если на отрезке функция имеет разные знаки, то где-то между
этими двумя точками существует хотя бы один корень функции)\\

Метод чем-то напоминает бинарный поиск. Суть его работы алгоритмически:
\begin{enumerate}
    \item Определяется промежуток, на котором требуется найти корень. Предполагается, что условия смены знака и непрерывности выполнены.
    \item Вычисляется середина отрезка с помощью формулы \( m = \frac{a + b}{2} \). 
    \item Если \( f(m) = 0 \), то корень найден. 
    \item Если \( f(m) \cdot f(a) < 0 \), то корень лежит в интервале \([a, m]\), и обновляем правую границу: \(b = m\).
    \item Если \( f(m) \cdot f(b) < 0 \), то корень лежит в интервале \([m, b]\), и обновляем левую границу: \(a = m\).
    \item Процесс повторяется, пока длина интервала не станет достаточно малой, что означает приближённое значение корня.
\end{enumerate}
\textit{В программе реализован ручной ввод пользователем математической функции и границ, в которых требуется изучить функцию и найти приблизительное значение экстремума.}
\vspace{1cm}

\textbf{Теоретическая подготовка, решение уравнения вида \textit{f(x) = 0}}\\ 
Требовалось реализовать программу, которая решит следующую задачу: 