\textbf{Этап реализации алгоритмов}
\vspace{1cm}

\textbf{Задание 1, метод Дихотомии}\\

Коротко про реализацию на выбранном языке программирования:
\\

Главная функция, поиск экстремума:

\begin{lstlisting}[style=JavaStyle, caption={Java Code for Dichotomy Method}]
public static float dichotomyMethod(String expression, float a, float b, float tolerance) {
    if (func(expression, a) * func(expression, b) > 0) {
        throw new IllegalArgumentException("f(a) and f(b) must have opposite signs!");
    }

    float c;
    XYSeries series = new XYSeries("Iterations");
    ArrayList<Float> intervalLengths = new ArrayList<>();

    while ((b - a) / 2.0 > tolerance) {
        c = (a + b) / 2;
        series.add(c, func(expression, c));

        intervalLengths.add((b - a));

        if (func(expression, c) == 0) {
            return c;
        } else if (func(expression, c) * func(expression, a) < 0) {
            b = c;
        } else {
            a = c;
        }
    }

    series.add((a + b) / 2, func(expression, (a + b) / 2));

    showGraph(series);

    plotIntervalLengths(intervalLengths);

    return (a + b) / 2;
}
    \end{lstlisting}

Определяется функция dichotomyMethod, которая принимает в себя следующие аргументы:
введенную пользователем функцию, правую и левую границы интересующей области, (float для
большей точности, чтобы не ограничиваться целыми числами) точность работы.\\

Согласно алгоритму, концы интересующего отрезка должны иметь противоположные знаки,
поэтому обрабатываем исключение вв противном случае.\\

XY серии и список с длинами интервала понадобится далее для построения графиков.\\

Тело функции - цикл. Пока находимся в рамках погрешности, введенной пользователем,
смотрим в середину отрезка, добавляем ее в список возможных экстремумов, также добавляем 
в свой список длину текущего интервала.\\
Далее все по алгоритму, если попали в ноль, то экстремум найден. Если нет, то оцениваем знаки
и сдвигаем границы области поиска.\\
В конце, когда вышли за пределы точности, добавляем последнюю найденную точку для отображения на графике,
возвращаем ее же как ответ и рисуем график со всеми промежуточными значениями, а также длинами интервалов при 
итерациях.
\begin{center}
    \textit{Полный код c пояснениями см. в приложениях.}
\end{center}
\vspace{1cm}

\textbf{Задание 2, решение уравнения}\\

\begin{lstlisting}[style=JavaStyle, caption={Java Code for Finding Multiple Roots}]
public static List<Double> findRoots(String function, double a, double b, double epsilon) {
    Expression expression = new ExpressionBuilder(function)
            .variables("x")
            .build();

    List<Double> roots = new ArrayList<>();
    double step = epsilon;

    for (double x = a; x <= b; x += step) {
        if (f(expression, x) * f(expression, x + step) < 0) {
            double root = findRootInInterval(expression, x, x + step, epsilon);
            roots.add(root);
        }
    }

    return roots;
}

private static double findRootInInterval(Expression expression, double a, double b, double epsilon) {
    if (f(expression, a) * f(expression, b) >= 0) {
        return Double.NaN;
    }

    double c = a;
    int iterations = 0;

    while ((b - a) >= epsilon) {
        c = (a + b) / 2;
        if (f(expression, c) == 0.0) {
            break;
        } else if (f(expression, c) * f(expression, a) < 0) {
            b = c;
        } else {
            a = c;
        }
        iterations++;
    }

    System.out.println("Number of iterations: " + iterations);
    return c;
}
    \end{lstlisting}
Для данного задания пришлось реализовать сразу две функции. Главную, которая будет выводить и
собирать все найденные корни и функцию поиска корня на каждом из отрезков. (вторая функция действует
по принципу дихотомии, пояснено в предыдущем задании)\\

В первой функции поиска всех корней инициализируем математическую функцию с помощью
библиотеки-билдера специально для этой задачи. ($ExpressionBuilder$)\\

Инициализируем список корней, инициализируем шаг размера $\epsilon$ для \textit{"передвижения"} по интервалу.\\

\begin{lstlisting}[style=JavaStyle, caption={Java Code, Main cycle}]
for (double x = a; x <= b; x += step) {
        if (f(expression, x) * f(expression, x + step) < 0) {
            double root = findRootInInterval(expression, x, x + step, epsilon);
            roots.add(root);
        }
    }
\end{lstlisting}
Далее в цикле: если знаки на концах различны, следовательно, есть корень,
вызываем функцию поиска корня на отрезке и заносим найденный корень в список корней.
Возвращаем список корней.\\
Задача выполнена.
\begin{center}
    \textit{Реализован вывод всех найденных корней и отклонения для каждого из них,\\
    полный код см. в приложении}
\end{center}
