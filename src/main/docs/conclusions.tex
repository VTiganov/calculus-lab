\textbf{Выводы}
\begin{enumerate}
    \item В ходе лабораторной работы мы изучили основные методы нахождения корней функций, такие как метод дихотомии.
    \item Метод дихотомии (или бисекции) — это простой и надежный способ найти корень функции. Он основан на теореме, которая говорит, что если непрерывная функция принимает значения разных знаков на концах интервала, то в этом интервале есть корень.
    \item Мы реализовали алгоритмы для нахождения корней функций на языке программирования Java. В частности, мы реализовали метод дихотомии, который постепенно сужает интервал, содержащий корень, до тех пор, пока его длина не станет меньше заданной точности.
    \item Мы провели эксперименты с различными функциями и интервалами, чтобы оценить точность и эффективность наших методов. Например, для функции \( f(x) = x^3 - 4x - 9 \) на интервале [2, 3] мы нашли корень с точностью до шести знаков после запятой.
    \item Мы провели тесты, чтобы проверить правильность наших методов. Мы проверяли случаи, когда функция не имеет корней в заданном интервале, а также случаи, когда функция имеет несколько корней.
    \item Результаты показали, что метод дихотомии надежен и прост в реализации, но может требовать больше итераций по сравнению с другими методами. Например, для функции \( f(x) = \sin(x) \) на интервале [3, 4] потребовалось около 20 итераций для достижения заданной точности.
    \item Важно правильно выбирать начальные условия и точность при использовании метода для нахождения корней, так как это может сильно влиять на результаты. Например, если начальный интервал выбран неправильно, метод может не сходиться к корню.
    \item В ходе лабораторной работы мы изучили и применили основные принципы численных методов, такие как итеративное приближение и оценка ошибки. Это помогло нам лучше понять, как работают численные методы и как их можно применять на практике.
\end{enumerate}